%%%%%%%%%%%%%%%%%%%%%%%%%%%%%%%%%%%%%%%%%
% Long Lined Cover Letter
% LaTeX Template
% Version 1.0 (1/6/13)
%
% This template has been downloaded from:
% http://www.LaTeXTemplates.com
%
% Original author:
% Matthew J. Miller
% http://www.matthewjmiller.net/howtos/customized-cover-letter-scripts/
%
% License:
% CC BY-NC-SA 3.0 (http://creativecommons.org/licenses/by-nc-sa/3.0/)
%
%%%%%%%%%%%%%%%%%%%%%%%%%%%%%%%%%%%%%%%%%

%----------------------------------------------------------------------------------------
%	PACKAGES AND OTHER DOCUMENT CONFIGURATIONS
%----------------------------------------------------------------------------------------

\documentclass[10pt,stdletter,dateno,sigleft]{newlfm} % Extra options: 'sigleft' for a left-aligned signature, 'stdletternofrom' to remove the from address, 'letterpaper' for US letter paper - consult the newlfm class manual for more options

\usepackage{charter} % Use the Charter font for the document text

\newsavebox{\Luiuc}\sbox{\Luiuc}{\parbox[b]{1.75in}{\vspace{0.5in}
\includegraphics[width=1.\linewidth]{USP.jpg}}} % Company/institution logo at the top left of the page
\makeletterhead{Uiuc}{\Lheader{\usebox{\Luiuc}}}

\newlfmP{sigsize=50pt} % Slightly decrease the height of the signature field
\newlfmP{addrfromphone} % Print a phone number under the sender's address
\newlfmP{addrfromemail} % Print an email address under the sender's address
\PhrPhone{Phone} % Customize the "Telephone" text
\PhrEmail{Email} % Customize the "E-mail" text

\lthUiuc % Print the company/institution logo

%----------------------------------------------------------------------------------------
%	YOUR NAME AND CONTACT INFORMATION
%----------------------------------------------------------------------------------------

\namefrom{\includegraphics[width=0.2\textwidth]{assinatura-2011-07-06__}\newline\newline Renato Fabbri} % Name

\addrfrom{
\today\\[12pt] % Date
%123 Broadway \\ % Address
%City, State 12345
University of S\~ao Paulo,\\
CP 369, 13560-970,\\
S\~ao Carlos, SP, Brazil
}

\phonefrom{+55 (16) 98107 7222} % Phone number

\emailfrom{fabbri@usp.br} % Email address

%----------------------------------------------------------------------------------------
%	ADDRESSEE AND GREETING/CLOSING
%----------------------------------------------------------------------------------------

\greetto{Dear PLOS ONE Editors,} % Greeting text
\closeline{Thank you for your consideration! \newline \newline Sincerely yours,} % Closing text

\nameto{}%PLOS ONE Editors} % Addressee of the letter above the to address

\addrto{
%PLOS Offices % To address
}

%----------------------------------------------------------------------------------------

\begin{document}
\begin{newlfm}

%----------------------------------------------------------------------------------------
%	LETTER CONTENT
%----------------------------------------------------------------------------------------

	Please find attached the manuscript entitled "Social Participation Ontology: community documentation, enhancements and use examples", which we submit for publication in the PLOS ONE Journal. The reasons why we chose PLOS ONE as the vehicle for our manuscript are as follows.
	In the manuscript we report the development of a participatory upper OWL ontolgy.
	We were unable to find similar conceptualization in literature, although consistent with other upper ontologies and characterizations of such social structures.
	The ontologic structures have had some uses in linked data sharing and conceptual maturing.
	Handpicked examples are the use by the Brazilian social participation portal for sharing linked data
	and the international process of first conceptualizations.
	This research article have been used as a reference by community but lacks a peer review chancel.
	All data, scripts and documentation are open since first versions, including the submited article.
	Processes are also kept open, with traces in git repositories, data repositories,
	dedicated portals, and specialized gadgets.
	Because the methods and results are multidisciplinary, compliant with open information standards and with aplications of public interest,
	we take the view that the manuscript may be of interest to a very wide readership.
	 We believe that NJP is therefore the ideal vehicle for this type of scientific contribution.
	 If
	 you feel that the manuscript is appropriate for your journal, we suggest the following reviewers: 
	 Tudor Groza,
	 Indra Neil Sarkar,                                                        and 
	 Chris Mavergames.

%
%	 \begin{figure}[!h]
%%		         \includegraphics[width=0.1\textwidth]{assinatura-2011-07-06_}
%		 \centering
%		         \includegraphics[width=0.1\textwidth]{assinatura-2011-07-06_}
%		 \end{figure}
%




%In addition to demonstrating the temporal stability and benchmarking, these results enable sound observation of both network and participant outliers and the consideration of the textual production of the networks, sectors and participants. Therefore, the document also presents a first consideration of typologies in such context, which bridges exact and human sciences. Indeed, the authors of the manuscript belong to different fields, and the principles adopted in the research comply with open science principles, as the email data are officially part of the GMANE database, scripts are official PyPI packages, data from other social networks and further scripts are in public Git repository. Furthermore, all the details of the theoretical developments and the results generated are given in a supplementary Supporting Information file, which will serve as guidance for anyone  interested in the field.
%	Because the methods applied belong to the realms of statistical physics (mainly) and since the subject is absolutely generic, we take the view that the manuscript may be of interest to a very wide readership. We believe that NJP is therefore the ideal vehicle for this type of scientific contribution.
%
%PARAGRAPH ONE: State the reason for the letter, name the position or type of work you are applying for and identify the source from which you learned of the opening (i.e. career development center, newspaper, employment service, personal contact).
%
%PARAGRAPH TWO: Indicate why you are interested in the position, the company, its products, services - above all, stress what you can do for the employer. If you are a recent graduate, explain how your academic background makes you a qualified candidate for the position. If you have practical work experience, point out specific achievements or unique qualifications. Try not to repeat the same information the reader will find in the resume. Refer the reader to the enclosed resume or application which summarizes your qualifications, training, and experiences. The purpose of this section is to strengthen your resume by providing details which bring your experiences to life. 
% 
%PARAGRAPH THREE: Request a personal interview and indicate your flexibility as to the time and place. Repeat your phone number in the letter and offer assistance to help in a speedy response. For example, state that you will be in the city where the company is located on a certain date and would like to set up an interview. Alternatively, state that you will call on a certain date to set up an interview. End the letter by thanking the employer for taking time to consider your credentials. 

%----------------------------------------------------------------------------------------
\end{newlfm}
\end{document}
